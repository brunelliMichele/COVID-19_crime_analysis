\documentclass[11pt,a4paper]{article}

% --- Packages ---
\usepackage[utf8]{inputenc}
\usepackage[T1]{fontenc}
\usepackage{lmodern}
\usepackage{geometry}
\geometry{margin=2.5cm}

\usepackage{graphicx}
\usepackage{adjustbox}
\usepackage{booktabs}
\usepackage{multirow}
\usepackage{amsmath}
\usepackage{hyperref}
\usepackage{longtable}
\usepackage[authoryear]{natbib}

% --- Metadata --
\title{Spatial patterns of reported crime in Italy before, during and after the COVID-19 pandemic}
\author{Michele Brunelli}
\date{\today}

\begin{document}
\maketitle

\begin{abstract}
This study examines the spatial patterns of reported crimes across Italian provinces over the period 2014--2023, with the aim of assessing whether the COVID-19 pandemic altered not only the intensity but also the geographic distribution of criminal activity. Crime data provided by Istat were normalised per 100,000 inhabitants and aggregated into three temporal phases: a pre-pandemic baseline (2014--2019), the acute pandemic period (2020--2021), and the post-pandemic recovery (2022--2023). The analysis was conducted at three nested geographic levels---macro-areas (NUTS-1), regions (NUTS-2), and provinces (NUTS-3)---using exploratory spatial analysis techniques, including global spatial autocorrelation (Moran's I) and local indicators of spatial association (LISA). Results reveal a clear dichotomy: contact-based crimes such as pickpocketing and residential burglary declined sharply, while digital offences including cybercrime and online fraud increased substantially. Spatial autocorrelation analysis shows that the pandemic fragmented the overall spatial structure of crime, though with heterogeneous effects across crime types. LISA cluster tracking identifies three distinct evolutionary trajectories: permanent geographic reorganisation for opportunity-dependent crimes, moderate spatial redistribution for digital offences, and high structural resilience for residential burglary. Sexual violence emerges as an anomalous case, displaying progressive spatial clustering intensification across all periods. A reproducible analysis pipeline and an interactive web application were developed to support the exploration of these patterns.
\end{abstract}

\section{Introduction}
% Introduction
The analysis of the spatial distribution of crime represents an important topic in applied geography and spatial data science, as criminal phenomena are intrinsically linked to the territorial, social, and institutional contexts in which they occur. Crime, in fact, is not randomly distributed in space, but tends to exhibit patterns of concentration, persistence, and clustering, shaped by socio-economic factors, urban structure, and the opportunities offered by different territorial contexts.

The COVID-19 pandemic constitutes a major shock that has profoundly altered everyday life, social interactions, and economic activities in recent years. Lockdown measures, mobility restrictions, and changes in people’s routines have likely affected both the overall volume of reported crimes and their spatial distribution. From a strictly geographical perspective, the pandemic can be interpreted as a natural experiment that allows observation of how crime patterns respond to sudden changes in human behaviour across space.

This project analyses the spatial patterns of reported crimes in Italy over the period 2014--2023, comparing three distinct phases: the pre-pandemic period (2014--2019), used as a baseline; the pandemic phase (2020--2021); and the recovery phase (2022--2023). The aim is to assess whether the observed changes in crime were homogeneous across the national territory or whether they assumed a spatially differentiated configuration, characterised by areas with specific local dynamics.

The analysis is conducted across three different geographical levels. A macro level, defined by five macro-areas dividing Italy into North-East, North-West, Centre, South, and Islands; a meso level, corresponding to the 20 Italian regions; and a micro level, at which the 107 provinces into which the country is divided are analysed. Given that Italy is characterised by territories with markedly different demographic sizes, crime rates are considered to achieve an appropriate result.

From a methodological perspective, the project adopts a reproducible analysis pipeline implemented in Python, integrating official statistical data with geographic boundaries and interactive visualisation tools. The analysis relies on exploratory spatial analysis techniques, including thematic cartography (choropleth maps) and local indicators of spatial association (LISA), aimed at identifying spatial patterns and clusters. To allow users to explore the distribution of crime across space and time, a web-based application has also been developed.

The objective of this work is not to establish causal relationships between the COVID-19 pandemic and crime trends, but rather to provide an exploratory and spatially explicit analysis of the transformations observed during a period of profound social disruption. The project therefore seeks to highlight both persistent—and pre-existing—territorial inequalities and local variations associated with the pandemic period.

\section{Data}
% Data Description

blabla \cite{sdmx}.

\section{Methodology}
% Methods
The analysis was conducted on administrative territorial units, using the NUTS classification across three nested spatial levels. The adoption of this spatial framework makes it possible to identify spatial patterns that may emerge, persist, or disappear across different geographical scales. Administrative units were selected because Istat provides crime statistics consistently at these levels, allowing for meaningful territorial comparisons. To account for the structural discontinuity introduced by the COVID-19 pandemic, crime data were aggregated into three distinct temporal periods. The first period represents the pre-pandemic phase (2014--2019) and is computed as a six-year average in order to reduce annual variability and obtain a robust baseline. The second period corresponds to the acute impact phase of the COVID-19 pandemic (2020--2021) and is calculated as a biennial average. The third and final period captures the post-pandemic recovery phase (2022--2023), also computed as a two-year average. This aggregation strategy enables a clear comparison between structurally different phases while limiting the influence of short-term fluctuations. The analysis relies on crime rates rather than absolute counts of reported offences. As shown in formula \ref{eq:rate}, crime rates are calculated as the number of reported crimes per 100,000 inhabitants, allowing comparability across territorial units and years.

\begin{equation}
\text{Rate}_{i,t} \;=\;
\frac{\text{Number of reported crimes}_{i,t}}{\text{Residente population}_{i,t}}\times 100{,}000
\label{eq:rate}
\end{equation}

The analysis is structured into three components, corresponding to the three analytical pages of the web application. The first page presents an interactive map showing the average variation in reported crimes relative to the baseline period. The map allows users to explore differences between the pandemic and post-pandemic phases, as well as variations across the three spatial levels (macro, meso, and micro). This descriptive and exploratory step makes it possible to observe how the intensity of criminal phenomena changes over time and across space. The map helps identify general patterns, geographical gradients, and areas characterized by marked increases or decreases. At this stage, the spatial structure of the phenomenon is not formally tested; rather, the analysis provides an interpretative foundation for subsequent spatial analyses.

The second page introduces an analysis of global spatial autocorrelation using Moran’s I. This index assesses whether observed crime rates follow a random spatial distribution or instead display systematic spatial clustering \citep{moran1950}. In other words, Moran’s I measures the degree of similarity between neighboring territorial units, allowing the identification of spatial concentration of high or low values. Applying this index to the data is particularly relevant because crime is an intrinsically spatial phenomenon and strongly dependent on territorial context. Examining Moran’s I across different temporal periods also allows for an assessment of whether the pandemic altered the overall level of spatial structuring of crime. This may occur, for example, through the attenuation of pre-existing patterns or the emergence of new ones as a consequence of mobility restrictions and changes in daily routines.

The final page deepens the analysis through local indicators of spatial autocorrelation (LISA). Unlike Moran’s I, LISA statistics \citep{anselin1995} enable the identification of local clusters and spatial outliers. This is achieved by distinguishing between areas characterized by high values surrounded by high values (\text{High-High}), low values surrounded by low values (\text{Low-Low}), and spatially dissimilar configurations. LISA is particularly useful because it allows for the precise localization of areas where crime tends to concentrate or disperse, highlighting intra-regional differences and dynamics that are not captured by global statistics. The use of LISA also makes it possible to assess whether the pandemic affected not only the intensity of criminal activity, but also the spatial location of crime clusters. This provides a more fine-grained interpretation of processes of territorial concentration and fragmentation. Local indicators of spatial autocorrelation (LISA) are therefore used to identify local clusters and spatial outliers. The analysis specifically aims to assess whether the pandemic modified the spatial distribution of crime clusters, with particular attention to contact-related crimes and digital offences. In doing so, it highlights potential processes of territorial concentration or spatial dispersion associated with the pandemic period.


\section{Results}
% Results
\subsection{Aggregate Crime Variation}

The analysis of average provincial-level variations reveals a distinctive pattern in the response of criminal phenomena to pandemic-related restrictions. As shown in Table~\ref{tab:crime_variation_mean}, the offences considered can be clearly divided into two groups according to the type of interaction required for their commission. Contact-based crimes experienced sharp contractions, with pickpocketing recording the most pronounced decline ($-52.4\%$). This finding is consistent with the drastic reduction in urban mobility and tourist flows observed during the lockdown period. Residential burglaries ($-44.0\%$) and snatch thefts ($-38.6\%$) display similar trends, suggesting a strong correlation with reduced circulation in public spaces and increased time spent at home.By contrast, digital crimes registered substantial increases. Offences related to the possession of child sexual abuse material ($+86.3\%$) and cybercrime ($+83.2\%$) nearly doubled relative to the baseline period. Online fraud increased by $67.2\%$, reflecting the intensification of digital activity during the pandemic. The widespread adoption of smart working, distance learning, and e-commerce significantly expanded the pool of potential targets for cybercriminal communities.An apparently anomalous pattern emerges from the analysis of sexual violence ($+16.8\%$). Although classified as a contact-based crime, this offence exhibits a significant increase during the pandemic period. This countertrend is likely related to the situation of survivors\footnote{The term ``survivor'' is used here in an analytical sense to refer to individuals who have experienced sexual violence, in order to emphasise the continuity of subjectivity beyond the traumatic event and to avoid defining individuals exclusively through victimisation.}, who were often forced to cohabit with their abusers during lockdowns. It is important to stress that the positive variation in this indicator is particularly concerning, especially when considering the well-documented reporting bias associated with sexual and domestic violence. Sociological literature highlights how survivors of domestic violence were especially vulnerable during lockdowns, being confined at home with abusers\footnote{The term ``abuser'' is used to refer to the perpetrators of sexual violence in an analytical sense, with the aim of highlighting the relational and behavioural dimensions of violence, rather than reducing these individuals solely to the legal category of ``offender'' or ``criminal''.} and facing reduced access to support services \citep{koss1992}.

\begin{table}[htbp]
\centering
\caption{Percentage variation in average crime rates during the COVID period (2020--2021) compared to the pre-pandemic baseline (2014-2019). Values calculated across 107 Italian provinces.}
\label{tab:crime_variation_mean}
\begin{tabular}{lllc}
\toprule
\textbf{Istat code} & \textbf{Crime Type} & \textbf{Group} & \textbf{Mean variation (\%)} \\
\midrule
PICKTHEF & Pickpocketing & Contact & $-52.4$ \\
BURGTHEF & Residential burglary & Contact & $-44.0$ \\
BAGTHEF & Snatch theft & Contact & $-38.6$ \\
STREETROB & Street robbery & Contact & $-17.2$ \\
RAPE & Sexual assault & Contact & $+16.8$ \\
\midrule
PORNO & Child sexual abuse material & Digital & $+86.3$ \\
CYBERCRIM & Cybercrime & Digital & $+83.2$ \\
SWINCYB & Online fraud & Digital & $+67.2$ \\
\midrule
TOT & Total reported crimes & --- & $-16.8$ \\
\bottomrule
\end{tabular}
\end{table}

Overall, the total number of reported crimes decreased by $16.8\%$. This aggregate reduction, however, masks a substantial displacement effect: criminal activity largely shifted from physical spaces to the digital domain, supporting the hypothesis of mobility-restriction-induced crime displacement. Aggregation at the national level conceals marked territorial heterogeneity. The analysis of provincial distributions reveals that variations were far from uniform across the Italian territory. As shown in Table~\ref{tab:provincial_variations}, the provinces recording the highest relative increases include medium-sized and small areas located both in the North (Mantua, Ferrara, Lecco) and in the South (Avellino and Caltanissetta). This pattern suggests that the forced digitalisation induced by the pandemic affected territories with very different socio-economic characteristics. It can be hypothesised that these crimes impacted large Italian cities to a lesser extent, where digitalisation was already more advanced compared to provincial areas; however, given the available data, this remains a tentative interpretation.

% Cybercrime growth table
\begin{table}[htbp]
\centering
\caption{Provinces with the largest percentage increases in cybercrime rates during COVID (2020--2021).}
\label{tab:cybercrime_increases}
\begin{tabular}{lccc}
\toprule
\textbf{Provinces} & \textbf{Variation (\%)} & \textbf{Baseline (per 100k)} & \textbf{During COVID (per 100k)} \\
\midrule
Mantova & $+334.9$ & 33.6 & 146.3 \\
Avellino & $+301.8$ & 12.5 & 50.3 \\
Ferrara & $+244.9$ & 10.6 & 36.5 \\
Caltanissetta & $+213.5$ & 10.7 & 33.5 \\
Lecco & $+195.6$ & 13.7 & 40.6 \\
\bottomrule
\end{tabular}
\end{table}

The offence showing the largest relative increase concerns the possession of child sexual abuse material ($+86.3\%$ at the national level). While this figure warrants particular attention due to the intrinsic severity of the phenomenon, it must be interpreted with methodological caution. As shown in Table~\ref{tab:csam_increases}, baseline rates are extremely low (often below one offence per 100{,}000 inhabitants), making percentage variations highly sensitive to small absolute changes. Nevertheless, despite low absolute values, the generalised increase observed across the majority of Italian provinces points to a systematic phenomenon rather than statistical noise. This pattern is consistent with two non-mutually exclusive mechanisms: (i) increased online exposure of minors during lockdowns, and (ii) intensified production and distribution facilitated by prolonged domestic confinement and increased digital traffic.

% CSAM growth table
\begin{table}[htbp]
\centering
\caption{Provinces with the largest percentage increases in child sexual abuse material (CSAM) offence rates during COVID (2020--2021).}
\label{tab:csam_increases}
\begin{tabular}{lccc}
\toprule
\textbf{Province} & \textbf{Variation (\%)} & \textbf{Baseline (per 100k)} & \textbf{During COVID (per 100k)} \\
\midrule
Valle d'Aosta & $+700.0$ & 0.4 & 3.2 \\
Novara & $+485.7$ & 0.35 & 2.05 \\
Bari & $+451.6$ & 0.52 & 2.85 \\
Nuoro & $+368.8$ & 0.53 & 2.5 \\
Medio Campidano & $+368.8$ & 0.53 & 2.5 \\
\bottomrule
\end{tabular}
\end{table}

The most pronounced reductions in pickpocketing rates are observed in provinces characterised by strong commercial and touristic specialisation (Table~\ref{tab:pickpocketing_decreases}). Trieste ($-83.0\%$), Enna ($-82.0\%$), and Prato ($-72.7\%$) exhibit particularly large declines, in line with their role as commercial and tourist hubs. Milan, while also recording a substantial decrease ($-34.7\%$), displays a contraction well below the national average ($-52.4\%$). This suggests that, even under restrictive conditions, the Lombard metropolis maintained a residual level of urban activity and mobility higher than that observed in other territorial contexts.

% Pickpocketing degrowth table
\begin{table}[htbp]
\centering
\caption{Provinces with the largest percentage decreases in pickpocketing rates during COVID (2020--2021).}
\begin{tabular}{lccc}
\toprule
\textbf{Province} & \textbf{Variation (\%)} & \textbf{Baseline (per 100k)} & \textbf{During COVID (per 100k)} \\
\midrule
Trieste & $-83.0$ & 425.7 & 72.2 \\
Enna & $-82.0$ & 30.1 & 5.4 \\
Prato & $-72.7$ & 260.8 & 71.3 \\
Campobasso & $-69.1$ & 59.2 & 18.3 \\
Potenza & $-68.9$ & 35.1 & 10.9 \\
\bottomrule
\end{tabular}
\end{table}

\subsection{Spatial Autocorreletion Analysis}

The analysis of spatial autocorrelation using the global Moran’s I index reveals differentiated patterns in the spatial structure of criminal phenomena across the three periods considered. Table~\ref{tab:moran_values} reports Moran’s I values for the selected crime categories.

% Moran's values
\begin{table}[htbp]
\centering
\small
\caption{Global Moran's I and p-values for selected crime types, calculated at provincial level (NUTS-3) across temporal periods.}
\label{tab:moran_values}
\begin{adjustbox}{max width=\textwidth}
\begin{tabular}{lcccccc}
\toprule
\multirow{2}{*}{\textbf{Crime type}} & \multicolumn{2}{c}{\textbf{Pre-COVID (2014--2019)}} & \multicolumn{2}{c}{\textbf{During COVID (2020--2021)}} & \multicolumn{2}{c}{\textbf{Post-COVID (2022--2023)}} \\
\cmidrule(lr){2-3} \cmidrule(lr){4-5} \cmidrule(lr){6-7}
& Moran's I & p-value & Moran's I & p-value & Moran's I & p-value \\
\midrule
Pickpocketing & 0.198 & 0.003 & 0.191 & 0.009 & 0.092 & 0.060 \\
Residential burglary & 0.729 & 0.001 & 0.625 & 0.001 & 0.682 & 0.001 \\
Sexual assault & 0.338 & 0.001 & 0.373 & 0.001 & 0.446 & 0.001 \\
Cybercrime & 0.371 & 0.001 & 0.245 & 0.002 & 0.272 & 0.002 \\
Total crimes & 0.253 & 0.001 & 0.123 & 0.023 & 0.149 & 0.020 \\
\bottomrule
\end{tabular}
\end{adjustbox}
\end{table}

Pickpocketing exhibits a moderate degree of spatial clustering in the pre-pandemic period ($I = 0.198$, $p$-value $= 0.003$). This pattern remains relatively stable during the COVID period ($I = 0.191$, $p$-value $= 0.009$), but it almost completely dissolves in the post-COVID phase ($I = 0.092$, $p$-value $= 0.060$), losing statistical significance. Moran scatter plots (Figure~\ref{fig:moran_pickpocketing}) visually confirm this fragmentation. In the pre-COVID period, observations are concentrated along the regression line, with clearly identifiable High--High and Low--Low clusters. This structure persists during the lockdown period. In the post-pandemic phase, however, points display a marked dispersion around the regression line, indicating that pre-existing spatial patterns did not re-emerge. This is likely attributable to the slow recovery of the tourism sector, which plays a crucial role in pickpocketing activity.

\begin{figure}[htbp]
\centering
\includegraphics[width=0.95\textwidth]{figures/pickpocketing.png}
\caption{Moran scatter plots for pickpocketing across three periods.}
\label{fig:moran_pickpocketing}
\end{figure}

Cybercrime shows relatively high spatial clustering in the pre-pandemic period ($I = 0.371$, $p$-value $= 0.001$). This clustering weakens substantially during the pandemic ($I = 0.245$, $p$-value $= 0.002$) and then stabilizes in the post-COVID phase ($I = 0.272$, $p$-value $= 0.002$). Moran scatter plots (Figure~\ref{fig:moran_cybercrime}) reveal a crucial detail. In the pre-COVID period, several extreme High--High outliers appear in the upper-right quadrant, indicating provinces with very high cybercrime rates surrounded by similarly high-rate neighbors. During the COVID period, these extreme outliers move closer to the regression line, suggesting a more geographically homogeneous distribution. This pattern is consistent with the hypothesis that forced digitalization reduced territorial disparities in exposure to cybercrime, extending digital vulnerabilities to areas that were previously less affected.

\begin{figure}[htbp]
\centering
\includegraphics[width=0.95\textwidth]{figures/cybercrimes.png}
\caption{Moran scatter plots for cybercrime across three periods.}
\label{fig:moran_cybercrime}
\end{figure}

Sexual violence exhibits an anomalous and progressive increase in spatial clustering across all three periods ($I = 0.338 \rightarrow 0.373 \rightarrow 0.446$, all with $p$-values $< 0.01$). Unlike other contact-based crimes, which show attenuation or fragmentation of spatial patterns, sexual violence displays increasing geographic concentration. Moran scatter plots (Figure~\ref{fig:moran_sexual_violence}) show an increasingly tight alignment of observations along the regression line, with a strengthening of High--High clusters in the post-COVID period. This pattern suggests that pandemic-related factors may have amplified pre-existing territorial inequalities in either the reporting or incidence of sexual violence. As discussed in Section~4.1, this increase may reflect the inability of survivors to leave their homes while being forced into prolonged close contact with their abusers.

\begin{figure}[htbp]
\centering
\includegraphics[width=0.95\textwidth]{figures/rape.png}
\caption{Moran scatter plots for sexual assault across three preiods.}
\label{fig:moran_rape}
\end{figure}

Finally, Moran’s I computed on the total number of reported crimes shows a substantial reduction in overall spatial clustering during the COVID period (from $I = 0.253$ in the pre-COVID phase to $I = 0.123$ during COVID, both with $p$-values $< 0.05$). Moran scatter plots (Figure~\ref{fig:moran_total}) display a clear dispersion of observations away from the regression line during the pandemic, indicating that COVID-19 fragmented the overall spatial structure of crime. In the post-COVID period, a partial recovery is observed ($I = 0.149$), but clustering remains below pre-pandemic levels, suggesting that the spatial equilibria of crime have not fully returned to their previous configuration.

\begin{figure}[htbp]
\centering
\includegraphics[width=0.95\textwidth]{figures/total.png}
\caption{Moran scatter plots for total reported crimes across three periods.}
\label{fig:moran_total}
\end{figure}

\section{Conclusions}
% Conclusions
\enlargethispage{\baselineskip}
The analysis of reported crimes across Italian provinces over the period 2014--2023 has shown that the COVID-19 pandemic produced spatially differentiated effects on the distribution of crime. This confirms that the pandemic shock did not act uniformly across the national territory.

The results reveal a clear dichotomy between contact crimes and digital crimes. The former---particularly pickpocketing ($-52.4\%$) and residential burglary ($-44.0\%$)---experienced sharp contractions during the pandemic period, consistent with the reduction in urban mobility and tourist flows. The latter---cybercrime ($+83.2\%$) and online fraud ($+67.2\%$)---registered substantial increases, reflecting the intensification of digital activity driven by remote work and distance learning. The aggregate decline of $16.8\%$ in total reported crimes therefore masks a displacement effect from the physical domain to the digital one. Spatial autocorrelation analysis revealed heterogeneous dynamics. The global Moran's I index showed a generalised fragmentation of the spatial structure during the pandemic, with reduced clustering for most crime categories. However, this effect was not uniform. Residential burglary maintained high spatial stability ($I = 0.729 \rightarrow 0.625 \rightarrow 0.682$), confirming the dominance of structural factors---housing typologies, urban density, and infrastructural accessibility---that pandemic restrictions did not substantially alter. By contrast, pickpocketing underwent a near-complete dissolution of pre-pandemic clusters, with a loss of statistical significance in the post-COVID period ($p$-value $= 0.060$). LISA analysis enabled the identification of three distinct evolutionary trajectories. Crimes dependent on transient opportunities, such as pickpocketing, underwent permanent geographic reorganisation. Stability rates fell below $10\%$, and clusters shifted from Ligurian coastal-tourist areas to the urban-industrial corridors of the Po Valley. Digital crimes exhibited moderate stability (${\sim}20\%$), with clusters migrating from Tuscany toward the North-West, likely linked to territorial differentials in digitalisation processes. Structural crimes, such as residential burglary, demonstrated high resilience (stability $> 35\%$), with clusters temporarily contracting during restrictions and subsequently re-expanding into geographically contiguous areas. A particularly noteworthy finding concerns sexual violence, which displays an anomalous trajectory compared to all other crime types analysed. Unlike other contact-based crimes, sexual violence shows a progressive increase in spatial clustering ($I = 0.338 \rightarrow 0.373 \rightarrow 0.446$) and an expansion of High--High clusters in the post-pandemic period, concentrated in Emilia-Romagna and Tuscany. This pattern is consistent with two non-mutually exclusive mechanisms. On the one hand, underreporting during lockdowns by survivors who were often cohabiting with their abusers. On the other hand, an actual increase in incidence driven by pandemic-related stressors such as social isolation, economic hardship, and prolonged domestic cohabitation.

This work presents several limitations that should be explicitly acknowledged. The annual granularity of Istat data does not allow the isolation of short-term dynamics, such as the specific effects of the March--May 2020 lockdown. The dataset includes only reported crimes, introducing a systematic bias linked to the differential propensity to report across social groups and territorial contexts. Furthermore, the analysis is exploratory in nature and does not permit the establishment of causal relationships between the pandemic and crime trends. Despite these limitations, the results suggest that the pandemic acted as a revealer of pre-existing territorial inequalities, amplifying certain dynamics and reconfiguring others. The methodological approach adopted---based on global and local spatial autocorrelation analysed across three temporal periods---proved effective in identifying patterns that are not observable through simple analysis of aggregate time series. Future developments could integrate explanatory socio-economic variables, adopt data with finer temporal granularity, and deepen the analysis of sexual violence by cross-referencing territorial data on anti-violence support services.

\bibliographystyle{apalike}
\bibliography{references}

\end{document}