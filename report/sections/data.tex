% Data Description
This section describes the data used for the analysis presented above. Official statistics on reported crimes were obtained by querying the SDMX APIs exposed via web services by Istat\footnote{Istat (Italian National Institute of Statistics) is the official statistical office of Italy. Official data are available at \url{https://www.istat.it/banche-dati/}.}. SDMX (Statistical Data and Metadata eXchange) is a standard widely adopted by statistical institutions to provide access to time series in a standardized and reproducible manner.

Data on crime rates are normilized per 100k inhabitants, and made available by Istat in several formats. For convenience, the data were downloaded in CSV format and subsequently aggregated---one dataset per year---into a single Parquet file. Parquet is a binary, columnar, and typed data format. This structure allows access to only the relevant columns without scanning the entire file each time, thereby minimizing input/output operations. Parquet is widely used in data analysis because it maximizes performance while minimizing both memory usage and storage requirements. Although the CSV files are described as containing annual data, they include observations for both the reference year and the subsequent year. For this reason, a consistency check was performed on the aggregated Parquet file in order to avoid duplicated values.

For map creation, Istat dataset were merged with geographic data provided by Eurostat GISCO\footnote{Eurostat GISCO (Geographic Information System of the Commission) provides harmonized geospatial data for the European Union. Administrative boundaries and related spatial datasets are available at \url{https://ec.europa.eu/eurostat/web/gisco}.}. Eurostat supplies geometries in GEOJSON format, organized by territorial levels. The levels used in this analysis are NUTS1 (macro-areas), NUTS2 (regions), and NUTS3 (provinces). Istat data rely on the 2006 version of the NUTS classification. As a result, it was necessary to manually map provinces that underwent administrative changes after 2006, most notably in Sardinia.

The data were aggregated into three time periods. The first corresponds to the pre-pandemic period (2014--2019) and reports the six-year average in order to obtain a robust baseline. The second period covers the years of acute pandemic impact (2020--2021) and reports their average. The third corresponds to the post-pandemic recovery phase and reports the average for the subsequent two years (2022--2023).

As mentioned above, the spatial analysis was conducted at three nested geographic levels. The macro level considers five areas of the Italian peninsula: North-East, North-West, Centre, South, and Islands. The meso level corresponds to regions, while the micro level corresponds to provinces.

The dataset containing crime rates classify offenses into 55 categories. Although the analytical pipeline and the web-based platform process all crime categories provided by Istat, the results discussed in this report focus on a subset of categories that are most relevant to the research question, as shown in Table~\ref{tab:crime_categories}. Crime categories were grouped based on the predominant modality of interaction involved in the offense, distinguishing between crimes requiring direct physical contact (\emph{contact crimes}) and crimes primarily mediated by digital infrastructures (\emph{digital crimes}). Based on changes in daily routines induced by the pandemic, a decrease in contact crimes is expected during the second analyzed period, while an increase in digital crimes is anticipated.

\begin{table}[htbp]
\centering
\caption{Crime categories discussed in the report (ISTAT codes)}
\label{tab:crime_categories}
\begin{tabular}{p{2.6cm}p{7.9cm}p{3.3cm}}
\toprule
\textbf{ISTAT code} & \textbf{Category (description)} & \textbf{Group} \\
\midrule
PICKTHEF  & Pickpocketing & Contact \\
BAGTHEF   & Snatch theft & Contact \\
BURGTHEF  & Residential burglary & Contact \\
STREETROB & Street robbery & Contact \\
RAPE      & Sexual assault & Contact \\
\midrule
SWINCYB   & Online fraud and cyber scams & Digital \\
CYBERCRIM & Cybercrime & Digital \\
PORNO     & Child sexual abuse material (CSAM) offences & Digital \\
\bottomrule
\end{tabular}
\end{table}

The dataset provided by Istat, however, presents several limitations that reduce the effectiveness of the analysis. Some of these issues could be addressed by improving data consistency, while others reflect structural biases\footnote{In this context, bias refers to a systemic distortion in the observed data that arises from the data genereation process itself, rather than from measurement or processing errors.} typical of large-scale administrative data. Annual data do not allow the identification of short-term or event-specific dynamics, but only the observation of broad trends. For instance, they do not make it possible to analyse in detail what occurred between March and May 2020, the period of full lockdown in Italy. The most significant source of bias concerns crime reporting. By construction, the dataset includes only crimes reported to the authorities. Sociological theory, however, shows that not all individuals are equally able to report victimisation \citep{biderman1967}. \citet{baumer2002} demonstrates that the propensity to report crimes varies according to several factors, including social class, socio-economic status, and the urban context in which individuals live. As a result, crimes experienced by socially marginalised groups are systematically underreported. This issue is particularly pronounced for crimes related to sexual violence \citep{koss1992} and child abuse. ``Survivors''\footnote{The term ``survivor'' is used here in an analytical sense to refer to individuals who have experienced sexual violence, in order to emphasise the continuity of subjectivity beyond the traumatic event and to avoid defining individuals exclusively through victimisation.} of these offences often live in the same physical environment as their abusers\footnote{The term ``abuser'' is used to refer to the perpetrators of sexual violence in an analytical sense, with the aim of highlighting the relational and behavioural dimensions of violence, rather than reducing these individuals solely to the legal category of ``offender'' or ``criminal''.}, which substantially increases the difficulty of reporting violence. It is therefore plausible that the incidence of such crimes increased significantly during the pandemic. However, the data may depict a different reality, driven by the practical impossibility of reporting abuse while sharing everyday spaces with the perpetrator.
