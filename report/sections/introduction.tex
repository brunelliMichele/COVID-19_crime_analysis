% Introduction
The analysis of the spatial distribution of crime represents an important topic in applied geography and spatial data science, as criminal phenomena are intrinsically linked to the territorial, social, and institutional contexts in which they occur. Crime, in fact, is not randomly distributed in space, but tends to exhibit patterns of concentration, persistence, and clustering, shaped by socio-economic factors, urban structure, and the opportunities offered by different territorial contexts.

The COVID-19 pandemic constitutes a major shock that has profoundly altered everyday life, social interactions, and economic activities in recent years. Lockdown measures, mobility restrictions, and changes in people’s routines have likely affected both the overall volume of reported crimes and their spatial distribution. From a strictly geographical perspective, the pandemic can be interpreted as a natural experiment that allows observation of how crime patterns respond to sudden changes in human behaviour across space.

This project analyses the spatial patterns of reported crimes in Italy over the period 2014--2023, comparing three distinct phases: the pre-pandemic period (2014--2019), used as a baseline; the pandemic phase (2020--2021); and the recovery phase (2022--2023). The aim is to assess whether the observed changes in crime were homogeneous across the national territory or whether they assumed a spatially differentiated configuration, characterised by areas with specific local dynamics.

The analysis is conducted across three different geographical levels. A macro level, defined by five macro-areas dividing Italy into North-East, North-West, Centre, South, and Islands; a meso level, corresponding to the 20 Italian regions; and a micro level, at which the 107 provinces into which the country is divided are analysed. Given that Italy is characterised by territories with markedly different demographic sizes, crime rates are considered to achieve an appropriate result.

From a methodological perspective, the project adopts a reproducible analysis pipeline implemented in Python, integrating official statistical data with geographic boundaries and interactive visualisation tools. The analysis relies on exploratory spatial analysis techniques, including thematic cartography (choropleth maps) and local indicators of spatial association (LISA), aimed at identifying spatial patterns and clusters. To allow users to explore the distribution of crime across space and time, a web-based application has also been developed.

The objective of this work is not to establish causal relationships between the COVID-19 pandemic and crime trends, but rather to provide an exploratory and spatially explicit analysis of the transformations observed during a period of profound social disruption. The project therefore seeks to highlight both persistent—and pre-existing—territorial inequalities and local variations associated with the pandemic period.