% Methods
The analysis was conducted on administrative territorial units, using the NUTS classification across three nested spatial levels. The adoption of this spatial framework makes it possible to identify spatial patterns that may emerge, persist, or disappear across different geographical scales. Administrative units were selected because Istat provides crime statistics consistently at these levels, allowing for meaningful territorial comparisons. To account for the structural discontinuity introduced by the COVID-19 pandemic, crime data were aggregated into three distinct temporal periods. The first period represents the pre-pandemic phase (2014--2019) and is computed as a six-year average in order to reduce annual variability and obtain a robust baseline. The second period corresponds to the acute impact phase of the COVID-19 pandemic (2020--2021) and is calculated as a biennial average. The third and final period captures the post-pandemic recovery phase (2022--2023), also computed as a two-year average. This aggregation strategy enables a clear comparison between structurally different phases while limiting the influence of short-term fluctuations. The analysis relies on crime rates rather than absolute counts of reported offences. As shown in formula \ref{eq:rate}, crime rates are calculated as the number of reported crimes per 100,000 inhabitants, allowing comparability across territorial units and years.

\begin{equation}
\text{Rate}_{i,t} \;=\;
\frac{\text{Number of reported crimes}_{i,t}}{\text{Residente population}_{i,t}}\times 100{,}000
\label{eq:rate}
\end{equation}

The analysis is structured into three components, corresponding to the three analytical pages of the web application. The first page presents an interactive map showing the average variation in reported crimes relative to the baseline period. The map allows users to explore differences between the pandemic and post-pandemic phases, as well as variations across the three spatial levels (macro, meso, and micro). This descriptive and exploratory step makes it possible to observe how the intensity of criminal phenomena changes over time and across space. The map helps identify general patterns, geographical gradients, and areas characterized by marked increases or decreases. At this stage, the spatial structure of the phenomenon is not formally tested; rather, the analysis provides an interpretative foundation for subsequent spatial analyses.

The second page introduces an analysis of global spatial autocorrelation using Moran’s I. This index assesses whether observed crime rates follow a random spatial distribution or instead display systematic spatial clustering \citep{moran1950}. In other words, Moran’s I measures the degree of similarity between neighboring territorial units, allowing the identification of spatial concentration of high or low values. Applying this index to the data is particularly relevant because crime is an intrinsically spatial phenomenon and strongly dependent on territorial context. Examining Moran’s I across different temporal periods also allows for an assessment of whether the pandemic altered the overall level of spatial structuring of crime. This may occur, for example, through the attenuation of pre-existing patterns or the emergence of new ones as a consequence of mobility restrictions and changes in daily routines.

The final page deepens the analysis through local indicators of spatial autocorrelation (LISA). Unlike Moran’s I, LISA statistics \citep{anselin1995} enable the identification of local clusters and spatial outliers. This is achieved by distinguishing between areas characterized by high values surrounded by high values (\text{High-High}), low values surrounded by low values (\text{Low-Low}), and spatially dissimilar configurations. LISA is particularly useful because it allows for the precise localization of areas where crime tends to concentrate or disperse, highlighting intra-regional differences and dynamics that are not captured by global statistics. The use of LISA also makes it possible to assess whether the pandemic affected not only the intensity of criminal activity, but also the spatial location of crime clusters. This provides a more fine-grained interpretation of processes of territorial concentration and fragmentation. Local indicators of spatial autocorrelation (LISA) are therefore used to identify local clusters and spatial outliers. The analysis specifically aims to assess whether the pandemic modified the spatial distribution of crime clusters, with particular attention to contact-related crimes and digital offences. In doing so, it highlights potential processes of territorial concentration or spatial dispersion associated with the pandemic period.
