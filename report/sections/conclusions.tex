% Conclusions
\enlargethispage{\baselineskip}
The analysis of reported crimes across Italian provinces over the period 2014--2023 has shown that the COVID-19 pandemic produced spatially differentiated effects on the distribution of crime. This confirms that the pandemic shock did not act uniformly across the national territory.

The results reveal a clear dichotomy between contact crimes and digital crimes. The former---particularly pickpocketing ($-52.4\%$) and residential burglary ($-44.0\%$)---experienced sharp contractions during the pandemic period, consistent with the reduction in urban mobility and tourist flows. The latter---cybercrime ($+83.2\%$) and online fraud ($+67.2\%$)---registered substantial increases, reflecting the intensification of digital activity driven by remote work and distance learning. The aggregate decline of $16.8\%$ in total reported crimes therefore masks a displacement effect from the physical domain to the digital one. Spatial autocorrelation analysis revealed heterogeneous dynamics. The global Moran's I index showed a generalised fragmentation of the spatial structure during the pandemic, with reduced clustering for most crime categories. However, this effect was not uniform. Residential burglary maintained high spatial stability ($I = 0.729 \rightarrow 0.625 \rightarrow 0.682$), confirming the dominance of structural factors---housing typologies, urban density, and infrastructural accessibility---that pandemic restrictions did not substantially alter. By contrast, pickpocketing underwent a near-complete dissolution of pre-pandemic clusters, with a loss of statistical significance in the post-COVID period ($p$-value $= 0.060$). LISA analysis enabled the identification of three distinct evolutionary trajectories. Crimes dependent on transient opportunities, such as pickpocketing, underwent permanent geographic reorganisation. Stability rates fell below $10\%$, and clusters shifted from Ligurian coastal-tourist areas to the urban-industrial corridors of the Po Valley. Digital crimes exhibited moderate stability (${\sim}20\%$), with clusters migrating from Tuscany toward the North-West, likely linked to territorial differentials in digitalisation processes. Structural crimes, such as residential burglary, demonstrated high resilience (stability $> 35\%$), with clusters temporarily contracting during restrictions and subsequently re-expanding into geographically contiguous areas. A particularly noteworthy finding concerns sexual violence, which displays an anomalous trajectory compared to all other crime types analysed. Unlike other contact-based crimes, sexual violence shows a progressive increase in spatial clustering ($I = 0.338 \rightarrow 0.373 \rightarrow 0.446$) and an expansion of High--High clusters in the post-pandemic period, concentrated in Emilia-Romagna and Tuscany. This pattern is consistent with two non-mutually exclusive mechanisms. On the one hand, underreporting during lockdowns by survivors who were often cohabiting with their abusers. On the other hand, an actual increase in incidence driven by pandemic-related stressors such as social isolation, economic hardship, and prolonged domestic cohabitation.

This work presents several limitations that should be explicitly acknowledged. The annual granularity of Istat data does not allow the isolation of short-term dynamics, such as the specific effects of the March--May 2020 lockdown. The dataset includes only reported crimes, introducing a systematic bias linked to the differential propensity to report across social groups and territorial contexts. Furthermore, the analysis is exploratory in nature and does not permit the establishment of causal relationships between the pandemic and crime trends. Despite these limitations, the results suggest that the pandemic acted as a revealer of pre-existing territorial inequalities, amplifying certain dynamics and reconfiguring others. The methodological approach adopted---based on global and local spatial autocorrelation analysed across three temporal periods---proved effective in identifying patterns that are not observable through simple analysis of aggregate time series. Future developments could integrate explanatory socio-economic variables, adopt data with finer temporal granularity, and deepen the analysis of sexual violence by cross-referencing territorial data on anti-violence support services.